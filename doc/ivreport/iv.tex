\documentclass[a4paper,12pt]{report}
\usepackage[latin1]{inputenc}
\usepackage[T1]{fontenc}
\usepackage[english]{babel}
\usepackage{float}
\usepackage{hyperref}
\usepackage{listings}
\usepackage{cite}
\usepackage{acro}
\usepackage{graphicx}
\usepackage[nodayofweek]{datetime}

\DeclareGraphicsRule{*}{mps}{*}{}

% Commands
\newcommand{\HRule}{\rule{\linewidth}{0.5mm}} % Defines a new command for the horizontal lines

\begin{document}

	\begin{titlepage}
		\begin{centering}
		 
		%	HEADING SECTIONS
		
		\textsc{\textbf{\LARGE{Industrial Visit Report}}}\\[0.5cm]

		\textbf{\textit{\large{A Project Report}}}\\[1.5cm]

		\large{submitted in partial fulfilment for the award of the degree of Bachelor of Technology in Computer Science and Engineering}\\[1.5cm]

		\large{by}\\[0.5cm]

		\textbf{Kevin Joseph     }\\
		{13400032 S7 R}\\[2cm]
		

		\includegraphics[width=5cm]{images/logo.jpg}

		\textsc{Department of Computer Science and Engineering}\\
		\textsc{College of Engineering Trivandrum}\\
		\textsc{Kerala}\\[0.5cm]
		\textsc{May 2017}\\
		\vfill % Fill the rest of the page with whitespace
		\end{centering}
	\end{titlepage}

	\begin{titlepage}
		\begin{centering}
			\textsc{\large{Department of Computer Science and Engineering}}\\
			\textsc{\large{College of Engineering Trivandrum}}\\[0.5cm]

			\includegraphics[width=5cm]{images/logo.jpg}\\[0.5cm]
			\textbf{\textit{\LARGE\textsc{{certificate}}}}\\[0.3cm]

		\end{centering}

		\begin{sloppypar}
		\large{This is to certify that this Industrial visit report is a bonafide record of the industrial visits undergone by Kevin Joseph (13400032), under our guidance towards partial fulfillment of the requirements for the award of Degree of bachelor of Technology in Computer Science and Engineering of the University of Kerala during the year 2016}\\[1.5cm]
		\end{sloppypar}

		\begin{minipage}{0.4\textwidth}
		\begin{flushleft}
		\begin{centering} \large
		\large{Mr. Sreelal S}\\
		\small{\textit{\textbf{Dept. of Computer Science and Engineering}}}\\[1.5cm]

		\end{centering}
		
		\end{flushleft}
		\end{minipage}
		~
		\begin{minipage}{0.5\textwidth}
		\begin{centering} \large
			\large{Mrs. Liji P I}\\
			\small{\textit{\textbf{Head of Department}}}\\
			\small{\textit{\textbf{Dept. of Computer Science and Engineering}}}\\
		\end{centering}
		\end{minipage}\\[1.0cm]

		\begin{flushleft}
		Place: Trivandrum\\
		Date:  11-05-2017\\
		\end{flushleft}
		\vfill % Fill the rest of the page with whitespace
	\end{titlepage}

	% TODO Ack before this
	\pagenumbering{roman}
	
	\newpage
	\tableofcontents
	\newpage

	\pagenumbering{arabic}
	\chapter{Sensomate Systems}
		\section{About The Company}
			\subsection{Technopark}
		\section{Products}
			\subsection{Proximity Beacons}
			\subsection{Attendance System}
			\subsection{SchoolSafe}
		\section{Software Development Lifecycle and Timeline}
		\section{Topics Discussed}
			\subsection{Data Analytics}
			Data analysis, also known as analysis of data or data analytics, is a process of inspecting, cleansing, transforming, and modeling data with the goal of discovering useful information, suggesting conclusions, and supporting decision-making. Data analysis has multiple facets and approaches, encompassing diverse techniques under a variety of names, in different business, science, and social science domains.

			Data analytics is a key area of focus for sensomate. All the major products of Sensomate Systems relies on Data analytics. The realtime attendance system Schoolsafe is the perfect example. It collects information about students in school and bus premises and stores and analyzes the data to take various decisions and display them to end users. The row data for this is the present/absent value for each student in regular intervals of time under each sensor. This data is stored in a centralized database and processed. The processing and querying is done by Influxdb, a time series database. The processed data can be used to determine whether a particular student was present in an area at a given time.
			\subsection{Realtime Web}
			The real-time web is a network web using technologies and practices that enable users to receive information as soon as it is published by its authors, rather than requiring that they or their software check a source periodically for updates. This is a major transition from old style websites where user just go to a website and refreshes the page everytime they want to get new updates. Realtime web is a huge area of development for Sensomate Systems. It is a vast area where a lot of new things happen everyday.

			The evolution of Javascript as a full stack development language accelerated the growth of realtime web. Realtime websites started to become a trend with the introduction of Ajax by Google. Ajax is a javascript library which can communicate with server side without refreshing the page. Jquery was also introduced as a easy-to-use and powerful library for javascript to attract a lot of companies in this area and it is still one of the most used javascript frameworks. The introducion of Node.js and AngularJs was a major milestone in Realtime web history. It was evolved into MEAN stack (Mongo Express Angular Node) and its the current choice of framework for most companies.

			In Sensomate, All realtime websites are developed using MEAN stack. The same can be developed as a mobile application using IONIC framework.
			\subsection{Mobile Application}
			\subsection{Internet of Things}
			\subsection{Content Managemant Systems}
			\subsection{Cloud Computing}
	
\end{document}